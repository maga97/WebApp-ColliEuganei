\documentclass[a4paper, 12pt]{report}
\usepackage[italian]{babel}
\usepackage[utf8]{inputenc}
\usepackage[T1]{fontenc}
\usepackage[scaled]{helvet}
\renewcommand\familydefault{\sfdefault}
\newcommand{\autori}{Costantino Marco\\Magarotto Francesco\\Piva Giulio\\ Zilio Davide}
\title{Progetto di Tecnologie Web\\ Colli Euganei}
\author{\autori}
\begin{document}
\maketitle
\begin{abstract}
Colli Digitali è un sito web informativo inerente i Colli Euganei in provincia
di Padova, che si propone di dare informazioni di carattere storico
relativamente alla formazione dei colli, paesi nella vicinanze e cosa visitare,
come ad esempio il Castello del Catajo, dimora storica della famiglia Obizzi nel
 1500. Inoltre, Colli Digitali permette agli utenti di effettuare prenotazioni
 per visitare il castello o fare un giro in bicicletta attraverso l'Anello dei
 Colli Euganei; tutte le attività sono comprensive di istruttore.
\end{abstract}
\section{Analisi dei requisiti}
\subsection{Back end}
L'applicazione web da sviluppare consiste in un sito informativo con l'aggiunta
di possibilità di prenotare visite/eventi presso il Castello del Catajo e/o gite
 in bicicletta tramite l'identificazione dell'utente. Oltre a ciò, è importante
permettere ai responsabili degli eventi di creare nuove visite/gite. \uppercase{è}
di fondamentale importanza l'implementazione di un sistema di \textit{newsletter},
che avvisi l'utente di nuovi eventi dopo la loro creazione da parte dei responsabili.
\subsection{Front end}
A lato front end è importante l'accessibilità e l'interazione con l'utente tramite
un'iterfaccia responsive che richiami i colori della natura, pertanto dev'essere
predominante il colore verde senza essere troppo invasivo non permettendo la
fruibilità dei contenuti.
\end{document}
